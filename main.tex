\documentclass{beamer}

\mode<presentation>
{
  \usetheme{default}      % or try Darmstadt, Madrid, Warsaw, ...
  \usecolortheme{default} % or try albatross, beaver, crane, ...
  \usefonttheme{default}  % or try serif, structurebold, ...
  \setbeamertemplate{navigation symbols}{}
  \setbeamertemplate{caption}[numbered]
  \setbeamertemplate{footline}[frame number]
} 

\hypersetup{colorlinks=true, urlcolor=blue}

\usepackage[english]{babel}
\usepackage[utf8x]{inputenc}

\title[2016-02-08-ROOT-JVM-firsttalk]{Accessing ROOT from the JVM (Java/Scala)}
\author{Jim Pivarski}
\date{2016-02-08}

\begin{document}

\begin{frame}
  \titlepage
\end{frame}

% Uncomment these lines for an automatically generated outline.
%\begin{frame}{Outline}
%  \tableofcontents
%\end{frame}

\begin{frame}{Motivation}
\begin{block}{}
Most of the big data-pipeline frameworks used in industry run on the Java Virtual Machine (JVM).
\end{block}

\begin{block}{}
\vspace{-\baselineskip}
In particular, Apache Spark is written in Scala.
\begin{itemize}
\item Scala is a JVM language (essentially interchangeable with Java, but more friendly for data analysis).
\item Spark supports analyses in Scala, Java, Python through sockets (Py4J), and R through pipes (stdin/stdout).
\item No support for C/C++ code, including ROOT.
\item Sockets and pipes both introduce serialization and transmission overhead.
\end{itemize}
\end{block}

\begin{block}{}
\vspace{-\baselineskip}
Similar motivation as for PyROOT: the JVM is a platform that is increasingly being used for data analysis.

\vspace{0.5\baselineskip}
We need an efficient and robust bridge.
\end{block}
\end{frame}

\begin{frame}{Technologies}

\begin{block}{}
Pure-Java reimplementation of ROOT I/O on \url{java.freehep.org} (with {\tt TTree} and 
\begin{itemize}
\item Hard to find: I thought it was gone. (User-facing documentation only has a JAR, compiled in 2001.)
\item But it lives! I found it in \url{svn://svn.freehep.org/svn/freehep/trunk} with commits as recent as Oct 2015 (last {\tt src/main} commit May 2014).
\item Reads and writes ROOT files with Java reflection to create runtime objects.
\end{itemize}
\end{block}






\end{frame}

%% No, the RIO reimplementation in JAS doesn't seem to exist anymore\ldots


\end{document}
