\documentclass{beamer}

\mode<presentation>
{
  \usetheme{default}      % or try Darmstadt, Madrid, Warsaw, ...
  \usecolortheme{default} % or try albatross, beaver, crane, ...
  \usefonttheme{default}  % or try serif, structurebold, ...
  \setbeamertemplate{navigation symbols}{}
  \setbeamertemplate{caption}[numbered]
  \setbeamertemplate{footline}[frame number]
} 

\usepackage[english]{babel}
\usepackage[utf8x]{inputenc}

\title[2016-02-08-ROOT-JVM-firsttalk]{Accessing ROOT from the JVM (Java/Scala)}
\author{Jim Pivarski}
\date{2016-02-08}

\begin{document}

\begin{frame}
  \titlepage
\end{frame}

% Uncomment these lines for an automatically generated outline.
%\begin{frame}{Outline}
%  \tableofcontents
%\end{frame}

\begin{frame}{Motivation}
\begin{block}{}
Most of the big data-pipeline frameworks used in industry run on the Java Virtual Machine (JVM).
\end{block}

\begin{block}{}
In particular, Apache Spark is written in Scala.
\begin{itemize}
\item Scala is a JVM language (essentially interchangeable with Java, but more friendly for data analysis).
\item Spark supports analyses in Scala, Java, (C)Python, and R.

Spark communicates with Python using Py4J (sockets) and R through pipes.



\end{itemize}
\end{block}





No, the RIO reimplementation in JAS doesn't seem to exist anymore\ldots

Similar to motivation for PyROOT; JVM is another platform that's starting to get a lot of analysis tools (Hadoop, Spark, and many others, mostly Apache stuff related to Big Data)
\end{frame}

\end{document}
